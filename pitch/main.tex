% Slide aspect ratio, theme and color theme
\documentclass[aspectratio=169]{beamer}
\usetheme{metropolis}
\usecolortheme[snowy]{owl}

% Modern font setup
\usepackage{fontspec}
\usepackage{unicode-math}

% Font selection
\setmainfont{Fira Sans}
\setsansfont{Fira Sans}
\setmonofont{Fira Code}
% \setmathfont{Fira Math}

% Language and layout
\usepackage[utf8]{inputenc}
\usepackage[brazil]{babel}

% Good to keep
\usepackage{amsmath}
\usepackage{graphicx}
\usepackage{url}
\usepackage{xcolor}
\usepackage{xstring}
\usepackage{outlines}

% Bibliography
\usepackage[backend=biber,style=authoryear]{biblatex}
\addbibresource{references.bib}
\renewcommand*{\nameyeardelim}{\addcomma\space}
\renewcommand*{\bibfont}{\color{black}\small}
\setbeamercolor{bibliography item}{fg=black}
\setbeamercolor{bibliography entry author}{fg=black}
\setbeamercolor{bibliography entry title}{fg=black}
\setbeamercolor{bibliography entry location}{fg=black}
\setbeamercolor{bibliography entry note}{fg=black}

% Wrapper for figures
\newcommand{\insertfigure}[3]{
  \StrBehind{#1}{/}[\FileWithExt]
  \StrBefore{\FileWithExt}{.}[\FileName]
  \begin{figure}[h!]
    \vspace{0.25cm}
    \centering
    \includegraphics[width=#3\linewidth]{#1}
    \caption{#2}
    \label{fig:\FileName}
  \end{figure}
}

% END PREAMBLE

\title{Sintetizando episódios de treino via abstração de jogos em Aprendizado por Reforço}
\author{Marcelo Augusto Salomão Ganem}
\institute{Departamento de Ciência da Computação \\ Universidade Federal de Minas Gerais}
\date{\today}
\begin{document}

\makeatletter

\begin{frame}
\titlepage
\end{frame}

\begin{frame}{Resumo}
    Este trabalho pretende verificar a efetividade da \textbf{sintetização de episódios utilizando modelagens abstrativas} dos jogos Blackjack e \textit{Settlers of Catan} seguindo a sintaxe \textit{Machinations} \parencite{machinations} no treino de agentes de aprendizado por reforço. 

    As políticas obtidas serão analisadas contra \textit{baselines} em implementações concretas dos jogos modelados para verificar a efetividade das intervenções propostas na velocidade e qualidade do aprendizado.

\end{frame}

% \begin{frame}{Outline}
% \tableofcontents
% \end{frame}


\section{Contexto}

\begin{frame}{Diagramas \textit{Machinations}}
        \begin{columns}
            \begin{column}{0.5\textwidth}
                \vspace{0.25cm}

                A sintaxe Machinations \parencite{machinations} introduz regras para a representação de jogos a uma variedade de níveis de abstração.
                \vspace{0.25cm}

                O modelo representa um jogo como um diagrama de nós e conexões de diversos tipos entre eles. Assim, o estado do jogo é a distribuição de \textbf{recursos} entre os nós.
            \end{column}
            \begin{column}{0.5\textwidth}
                \insertfigure
                    {figures/catan.png}
                    {Uma representação do jogo \textit{Settlers of Catan} sob a sintaxe de Machinations}
                    {0.8}
            \end{column}
        \end{columns}
    
\end{frame}

\begin{frame}{Diagramas \textit{Machinations}}
    \begin{figure}[h!]
        \begin{columns}
            \begin{column}{0.6\textwidth}
                \vspace{0.5cm}
                \includegraphics[width=1.0\linewidth]{figures/nodes.png}
            \end{column}
            \begin{column}{0.4\textwidth}
                \caption{Trecho do Apêndice A de \textit{Engineering Emergence} \parencite{machinations} descrevendo elementos básicos dos diagramas}
            \end{column}
        \end{columns}
        \label{fig:nodes}
    \end{figure}
\end{frame}

\begin{frame}{Diagramas \textit{Machinations}}
    \begin{columns}
        \begin{column}{0.6\textwidth}
            Bla bla bla 
        \end{column}
        \begin{column}{0.4\textwidth}
            \insertfigure
              {figures/monopoly.png}
              {\textit{Monopoly} (\citeyear{monopoly}) representado como um diagrama \textit{Machinations} \parencite{machinations}.}
              {1.0}
        \end{column}
    \end{columns}
\end{frame}


\begin{frame}{Diagramas \textit{Machinations}}
    As regras definidas por \citeauthor{machinations} não são formais o suficiente para permitir implementação direta\footnote{À época da publicação, existia uma simulação dinâmica disponível para o público -- hoje o serviço é prestado como B2B pela empresa Machinations, sediada na Holanda}. Assim, parte desse trabalho é transcrever a definição descritiva para uma definição formal.
\end{frame}

\section{Objetivos}

\begin{frame}{Objetivos}
    Utilizando diagramas Machinations como base, pretende-se \textbf{acelerar o treino de agentes de aprendizado por reforço via episódios sintetizados} como objetivo geral. Entendendo os diagramas como abstrações de problemas de aprendizado por reforço, o trabalho endereça especificamente as seguintes perguntas de pesquisa:
    \begin{outline}
        \1 Agentes treinados \textit{somente} na abstração produzem políticas coerentes no problema real?
        \1 É possível extrair representações úteis explorando a abstração?
    \end{outline}
\end{frame}

\section{Metodologia}

\begin{frame}{Metodologia}
    Overview da metodologia
    \begin{outline}
        \1 Objetivo geral
            \2 Objetivo específico 1
            \2 Objetivo específico 2
    \end{outline}
\end{frame}

\begin{frame}{Formalizando diagramas \textit{Machinations}}
    "\textbf{Nós} são representados como vértices $v \in V$, seu estado $x_{v,r}: V \times R \rightarrow \mathbb{R}$ (o número de recursos do tipo $r$ no nó $v$) e a função $m_{v}: V \rightarrow \{0,1,2\}$ como o modo de ativação do nó entre passivo, automático ou interativo [...]"
\end{frame}

\begin{frame}{Formalizando diagramas \textit{Machinations}}

    \begin{columns}
        \begin{column}{0.4\textwidth}
            \vspace{0.75cm}

            \textit{"Triggers fire when all the inputs of its source node become satisfied: when each input passed the number of resources to the node as indicated by its flow rate"}
        \end{column}
        \begin{column}{0.1\textwidth}
            \begin{center}
                $\rightarrow$
            \end{center}
        \end{column}
        \begin{column}{0.4\textwidth}
            \vspace{0.75cm}
            \begin{align*}
                &\{v: (u \rightarrow v) \in E^G\ | \\\ &\Big(\prod_{e \in E^R | e = (* \rightarrow u)}{\Delta e(t-1) = T_e \Big)} = 1 \}\\
            \end{align*}
        \end{column}
    \end{columns}
\end{frame}

\begin{frame}{Definindo o POMDP}
    Bla bla bla...
\end{frame}

\begin{frame}{Definindo o POMDP}
    Bla bla bla...
\end{frame}

\begin{frame}{Definindo o POMDP}
    Bla bla bla...
\end{frame}

\begin{frame}{Interface comum}
    Simular episódios com machinations de maneira a mostrar a mesma "interface" pro agentque uma implementação do jogo de verdade. Executar episódios de treino mais rapidamente.
\end{frame}

\begin{frame}{Implementação}
    Simulação, Gym wrapper, Clone dos Jogos, algoritmos de RL, visualizações com Manim.
\end{frame}

\begin{frame}{Plano de experimentos}
    Experimentos...
    \begin{outline}
        \1 Objetivo geral
            \2 Objetivo específico 1
            \2 Objetivo específico 2
    \end{outline}
\end{frame}

\begin{frame}
    \printbibliography
\end{frame}


\end{document}

