\documentclass[10pt,a4paper]{article}

% Modern font setup
\usepackage{fontspec}
\usepackage{unicode-math}

% Font selection
\setmainfont{Fira Sans}
\setsansfont{Fira Sans}
\setmonofont{Fira Code}

% Language and layout
\usepackage[utf8]{inputenc} % optional with XeLaTeX, but safe
\usepackage[brazil]{babel}
\usepackage{geometry}
\geometry{margin=2.5cm}

% Good to keep
\usepackage{amsmath}
\usepackage{graphicx}
\usepackage{url}
\usepackage{xcolor}
\usepackage{outlines}

% Bibliography
\usepackage[backend=biber,style=authoryear]{biblatex}
\addbibresource{references.bib}
\renewcommand*{\nameyeardelim}{\addcomma\space}

\title{Estruturação de Estados orientada a Game Design em Aprendizado por Reforço}
\author{{\bfseries Marcelo Augusto Salomão Ganem }\\ Universidade Federal de Minas Gerais\\
\texttt{marceloganem@dcc.ufmg.br}
}
\date{\today}
\newcommand{\note}[1]{
    \vspace{0.3cm}
    \colorbox{blue!30}{
            \begin{minipage}{0.4\textwidth}
		    \ttfamily \footnotesize
               #1
            \end{minipage}
        }
    \vspace{0.3cm}
}

\setlength{\columnsep}{1cm} % or whatever you prefer
\begin{document}

\makeatletter
% Title page, optional
% \begin{titlepage}
  % \centering
  % \vspace*{5.5cm}
    % {\Large \bfseries Above title \par}
    % \vspace{1cm}
    % {\Huge \bfseries \@title \par}
  % \vspace{1cm}
  % \vspace{0.5cm}
  % {\Large \@author \par}
  % \vspace{0.5cm}
  % \vfill {\large \today\par}
% \end{titlepage}

\twocolumn
\maketitle
\begin{abstract}
    this is the abstract
\end{abstract}


\section{Introdução}
\label{introduction}

\section{Objetivos}
\label{goals}

Tem-se por objetivo geral deste trabalho rq

\begin{outline}

    \1 rq1

    \1 rq2

\end{outline}

Dessa maneira, espera-se ...

\section{Referencial Teórico}
\label{bw}

\subsection{Machinations}
\label{bw:machinations}

\section{Trabalhos Relacionados}
\label{rw}

\section{Metodologia}
\label{m}

\subsection{Modelagem da sintaxe Machinations}
\label{m:modeling}
Os elementos descritos na seção \ref{bw:machinations} são definidos por \citeauthor{machinations} de maneira formal o suficiente a permitir uma simulação, mas não exatamente adequada para a implementação de um MDP no contexto de aprendizado por reforço. Assim, essa subseção se dedica à redefinição formal -- a algum nível de abstração -- dos elementos que compõem um diagrama de Machinations.

\subsubsection{Nós}
\textbf{Pools}, conforme descrito na seção \ref{bw:machinations}, são representadas como vértices $v \in V$. Definimos sua função de estado $x_{v,r}: V \times R \rightarrow \mathbb{R}$ como o número de recursos do tipo $r$ no nó $v$ e a função $m_{v}: V \rightarrow \{0,1,2\}$ como o modo de operação do nó entre passivo, automático ou interativo.

\textbf{Gates} são (...)

\subsubsection{Conexões}
Conexões são representadas como arestas $e \in E$, tal que $E$ é a união de uma série de subconjuntos que representam cada um dos diferentes tipos de conexões descritas na seção \ref{bw:machinations}.

\begin{align*}
    E &= E^R \cup E^S\\
    E^R &= \bigcup_{r \in R} E^r\\
    E^S &= E^{T} \cup E^{N} \cup E^{G} \cup E^{A}
\end{align*}

Essa composição, apesar de extensa, define toda a taxonomia das conexões definidas por \citeauthor{machinations}.

\textbf{Conexões de recursos} são representadas pelo subconjunto $E^R$ como a união das conexões para cada tipo de recurso. Uma conexão de recurso específica $e \in E^R$ é da forma $(u \rightarrow v)$ e tem definida para si uma função de taxa $T_e$ --- uma variável aleatória\footnote{Para lidar com fluxos inteiros, reais ou probabilísticos} representando o fluxo de recursos do nó $u$ para o nó $v$ em um instante de tempo.

O conjunto $E^S$ define as \textbf{conexões de estado}. Essas, por sua vez, se dividem em:

\begin{outline}
    \1 $E^T$: \textbf{modificadores de rótulo}\footnote{Do original \textit{label modifiers} \parencite{machinations}. O rótulo aqui é representado pela taxa $T_e$}, $e \in E^T = (u, r\rightarrow e')$, tal que a taxa $\dot{T}_{u,r,e}$ representa como mudanças no estado $x_{u, r}$ alteram o valor de $T_{e'}$.
    \1 $E^N$: \textbf{modificadores de nó}, $e \in E^N = (u, r \rightarrow v)$. A função $\dot{T}_{u,r,v}$ semelhantemente define como mudanças no estado $x_{u, r}$ alteram o valor de $x_{v, r}$.
    \1 $E^G$: \textbf{gatilhos}, $e \in E^G = (u, v)$ tal que (se recursos entraram por todos os inputs de $u$) (então adicione $v$ a $V^*$ que são os nós ativos)
    A node that has no inputs will fire outgoing triggers whenever it fires
    \1 $E^A$: \textbf{ativadores} They activate or inhibit their target node
based on the state of their source node and a specific condition.  Conditions are written as an arithmetic expression (for example ‘==0’, ‘<3’, ‘>=4’ or ‘!=2’) or a range of values (for example ‘3-6’). If the state of the source node meets this condition then the target node is activated (it can fire). When the condition is not met the target node is inhibited (it cannot fire).
    \note{I think we can just infer whether a node is activatable from the current state of activators and other nodes}
\end{outline}

\subsubsection{Dinâmica}


\subsection{Implementação do problema}
\label{m:mdp}
Observabilidade parcial...

\subsection{Plano de Experimentos}
\label{m:experiments}

\printbibliography


\end{document}

