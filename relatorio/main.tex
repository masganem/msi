\documentclass[10pt,a4paper]{article}

% Modern font setup
\usepackage{fontspec}
\usepackage{unicode-math}

% Font selection
\setmainfont{Fira Sans}
\setsansfont{Fira Sans}
\setmonofont{Fira Code}

% Language and layout
\usepackage[utf8]{inputenc} % optional with XeLaTeX, but safe
\usepackage[brazil]{babel}
\usepackage{geometry}
\geometry{margin=2.5cm}

% Good to keep
\usepackage{amsmath}
\usepackage{graphicx}
\usepackage{url}
\usepackage{xcolor}
\usepackage{outlines}

% Bibliography
\usepackage[backend=biber,style=authoryear]{biblatex}
\addbibresource{references.bib}
\renewcommand*{\nameyeardelim}{\addcomma\space}

\title{Estruturação de Estados orientada a Game Design em Aprendizado por Reforço}
\author{{\bfseries Marcelo Augusto Salomão Ganem }\\ Universidade Federal de Minas Gerais\\
\texttt{marceloganem@dcc.ufmg.br}
}
\date{\today}
\newcommand{\note}[1]{
    \vspace{0.3cm}
    \colorbox{blue!30}{
            \begin{minipage}{0.4\textwidth}
		    \ttfamily \footnotesize
               #1
            \end{minipage}
        }
    \vspace{0.3cm}
}

\setlength{\columnsep}{1cm} % or whatever you prefer
\begin{document}

\makeatletter
% Title page, optional
% \begin{titlepage}
  % \centering
  % \vspace*{5.5cm}
    % {\Large \bfseries Above title \par}
    % \vspace{1cm}
    % {\Huge \bfseries \@title \par}
  % \vspace{1cm}
  % \vspace{0.5cm}
  % {\Large \@author \par}
  % \vspace{0.5cm}
  % \vfill {\large \today\par}
% \end{titlepage}

\twocolumn
\maketitle
\begin{abstract}
    this is the abstract
\end{abstract}


\section{Introdução}
\label{introduction}

\section{Objetivos}
\label{goals}

Tem-se por objetivo geral deste trabalho rq

\begin{outline}

    \1 rq1

    \1 rq2

\end{outline}

Dessa maneira, espera-se ...

\section{Referencial Teórico}
\label{bw}

\subsection{Machinations}
\label{bw:machinations}

\section{Trabalhos Relacionados}
\label{rw}

\section{Metodologia}
\label{m}

\subsection{Modelagem da sintaxe Machinations}
\label{m:modeling}
Os elementos descritos na seção \ref{bw:machinations} são definidos por \citeauthor{machinations} de maneira formal o suficiente a permitir uma simulação, mas não exatamente adequada para a implementação de um MDP no contexto de aprendizado por reforço. Assim, essa subseção se dedica à redefinição formal -- a algum nível de abstração -- dos elementos que compõem um diagrama de Machinations.

\subsubsection{Nós}
\textbf{Nós}, conforme descrito na seção \ref{bw:machinations}, são representados como vértices $v \in V$. Definimos sua função de estado $x_{v,r}: V \times R \rightarrow \mathbb{R}$ como o número de recursos do tipo $r$ no nó $v$ e a função $m_{v}: V \rightarrow \{0,1,2\}$ como o modo de ativação do nó entre passivo, automático ou interativo. Nós interativos acionados no instante $t$ são dicionados ao conjunto $V^+(t)$.

\textbf{Pools}, portanto, são nós que acumulam recursos e sem propriedades especiais. \textbf{Gates} são semelhantes às pools e também são nós $v \in V$, porém imediatamente redistribuem recursos conforme o seu modo de distribuição $d_v : V \rightarrow \{0,1\}$ -- determinístico ou aleatório, causando a distribuição, duplicação ou destruição de recursos conforme descrito na Subseção \ref{m:modeling:dynamics}.

\subsubsection{Conexões}
\label{m:modeling:connections}
Conexões são representadas como arestas $e \in E$, tal que $E$ é a união de uma série de subconjuntos que representam cada um dos diferentes tipos de conexões descritas na seção \ref{bw:machinations}.

\begin{align*}
    E &= E^R \cup E^S\\
    E^R &= \bigcup_{r \in R} E^r\\
    E^S &= E^{T} \cup E^{N} \cup E^{G} \cup E^{A}
\end{align*}

Essa composição, apesar de extensa, define toda a taxonomia das conexões definidas por \citeauthor{machinations}.

\textbf{Conexões de recursos} são representadas pelo subconjunto $E^R$ como a união das conexões para cada tipo de recurso $r$. Uma conexão de recurso específica $e \in E^R$ é da forma $(u \rightarrow v)$ e tem definida para si uma função de taxa $T_e$ --- uma variável aleatória\footnote{Para lidar com fluxos inteiros, reais ou probabilísticos} representando o fluxo de recursos do nó $u$ para o nó $v$ em um instante de tempo. O tipo de recurso a ser transferido por $e$ é identificado por $\tau (e) = r$.

O conjunto $E^S$ define as \textbf{conexões de estado}. Essas, por sua vez, se dividem em:

\begin{outline}
    \1 $E^T$: \textbf{modificadores de taxa}\footnote{Do original \textit{label modifiers} \parencite{machinations}. O rótulo aqui é representado pela taxa $T_e$}, $e \in E^T = (u, r\rightarrow e')$, tal que a taxa $\dot{T}_{u,r,e}$ representa como mudanças no estado $x_{u, r}$ alteram o valor de $T_{e'}$.
    \1 $E^N$: \textbf{modificadores de nó}, $e \in E^N = (u, r \rightarrow v)$. A função $\dot{T}_{u,r,v}$ semelhantemente define como mudanças no estado $x_{u, r}$ alteram o valor de $x_{v, r}$.
    \1 $E^G$: \textbf{gatilhos}, $e \in E^G = (u \rightarrow v)$, seguem a definição da Seção \ref{bw:machinations} e determinam se um nó faz parte ou não do conjunto $V^*(t)$ de nós ativos\footnote{Da definição original \textit{firing} \parencite{machinations}} no tempo $t$. 
    \1 $E^A$: \textbf{ativadores}, $e \in E^A = (u \rightarrow v)$, determinam uma condição $P_e$ em função do estado $x_{u}$ tal que, quando não satisfeita no instante $t$, $v \notin V^*(t)$.
    % They activate or inhibit their target node based on the state of their source node and a specific condition.  Conditions are written as an arithmetic expression (for example ‘==0’, ‘<3’, ‘>=4’ or ‘!=2’) or a range of values (for example ‘3-6’). If the state of the source node meets this condition then the target node is activated (it can fire). When the condition is not met the target node is inhibited (it cannot fire).\\
    % \note{I think we can just infer whether a node is activatable from the current state of activators and other nodes}
\end{outline}

\subsubsection{Dinâmica}
\label{m:modeling:dynamics}
A Seção 4.2 de \textit{Engineering Emergence} define três modos de execução temporal -- este trabalho se limita ao caso de tempo síncrono onde, a cada instante $t$, todos os nós automáticos\footnote{Contadores podem ser implementados como um nó adicional e suas respectivas conexões} e os nós interativos acionados no instante $t-1$ são ativados. Conforme a definição original, todos os nós ativos no instante $t$ são processados ao mesmo tempo, levando a condições de corrida desambiguadas com regras específicas onde pertinente.

A transição de estados entre o tempo $t$ e o tempo $t+1$ dadas as definições acima pode ser computada primeiro identificando quais elementos de $V$ estão em $V^*(t)$ e então distribuindo os recursos conforme as conexões $e$ pertinentes a quaisquer $u,v \in V^*$.

Para definir os nós pertencentes a $V^*(t)$, consideramos nós automáticos, nós interativos acionados na etapa anterior, nós ativados por gatilhos e o efeito de ativadores:

\begin{align*}
    V^*(t) &= \{v \in V \ | \ m_v = 1\}\\& \cup \{v \in V^+(t-1)\}\\
    & \cup \{v: (u \rightarrow v) \in E^G\ | \\\
    &(\prod_{e \in E^R | e = (* \rightarrow u)}{\Delta e(t-1) = T_e)} = 1 \}\\
    &\cap \{v: (u \rightarrow v) \in E^A\ | \ P_e(x_u, t) = 1 \}\\
\end{align*}

Então, tomamos as arestas partindo de cada nó $u$ ativo no instante atual $E^*(t) = \{e \in E, u_e \in V^*(t)\}$ como o conjunto das conexões a serem disparadas no instante $t$. Tomando a interseção com cada um dos subconjuntos de $E$ definidos nessa seção, podemos atualizar o estado de cada nó ou conexão pertinente. Denominamos $E^{T*}$ o conjunto $E^T \cap E^*$, e assim equivalentemente para os demais subconjuntos definidos na Subseção \ref{m:modeling:connections}.

Atualizamos primeiro os alvos dos modificadores de taxa -- para cada alvo $e$ no conjunto \{$e \in E^R \ | \ (u, r \rightarrow e) \in E^{T*}\}$, temos:

$$
T_{e}(t+1) = T_{e}(t) + \dot{T}_{u,r,e} \Delta x_{u,r}
$$

Em seguida, atualizamos o estado dos alvos de modificadores de nó. Para cada nó $v$ no conjunto $\{v \in V \ | \ (u, r \rightarrow v) \in E^{N*}\}$:

$$
x_{v, r}(t+1) = x_{v, r}(t) + \dot{T}_{u,r,v} \Delta x_{u,r}
$$

Por fim, fazemos a transferência de recursos conforme cada $e \in E^{R*}$, sendo $(u, v) = e$ e $r = \tau(e)$:

\begin{align*}
    x_{v, r}(t+1) &= x_{v,r}(t) + T_e(t)\\
    x_{u, r}(t+1) &= x_{u,r}(t) - T_e(t)\\
\end{align*}

Esta descrição da dinâmica permite a simulação completa de diagramas Machinations como um ambiente de MDP de estado $S = \{x_v : v \in V\}$, garantidas desambiguações na implementação.

\subsection{Implementação}
\label{m:mdp}
Observabilidade parcial...

\subsection{Plano de Experimentos}
\label{m:experiments}

\printbibliography


\end{document}

