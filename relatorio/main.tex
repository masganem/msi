\documentclass[10pt,a4paper]{article}

% Modern font setup
\usepackage{fontspec}
\usepackage{unicode-math}

% Font selection
\setmainfont{Fira Sans}
\setsansfont{Fira Sans}
\setmonofont{Fira Code}

% Language and layout
\usepackage[utf8]{inputenc} % optional with XeLaTeX, but safe
\usepackage[brazil]{babel}
\usepackage{geometry}
\geometry{margin=2.5cm}

% Good to keep
\usepackage{amsmath}
\usepackage{graphicx}
\usepackage{url}
\usepackage{xcolor}
\usepackage{outlines}

% Bibliography
\usepackage[backend=biber,style=authoryear]{biblatex}
\addbibresource{references.bib}
\renewcommand*{\nameyeardelim}{\addcomma\space}

\title{Estruturação de Estados orientada a Game Design em Aprendizado por Reforço}
\author{{\bfseries Marcelo Augusto Salomão Ganem }\\ Universidade Federal de Minas Gerais\\
\texttt{marceloganem@dcc.ufmg.br}
}
\date{\today}
\newcommand{\note}[1]{
    \vspace{0.3cm}
    \colorbox{blue!30}{
            \begin{minipage}{0.4\textwidth}
		    \ttfamily \footnotesize
               #1
            \end{minipage}
        }
    \vspace{0.3cm}
}

\setlength{\columnsep}{1cm} % or whatever you prefer
\begin{document}

\makeatletter
% Title page, optional
% \begin{titlepage}
  % \centering
  % \vspace*{5.5cm}
    % {\Large \bfseries Above title \par}
    % \vspace{1cm}
    % {\Huge \bfseries \@title \par}
  % \vspace{1cm}
  % \vspace{0.5cm}
  % {\Large \@author \par}
  % \vspace{0.5cm}
  % \vfill {\large \today\par}
% \end{titlepage}

\twocolumn
\maketitle
\begin{abstract}
    this is the abstract
\end{abstract}


\section{Introdução}
\label{introduction}

\section{Objetivos}
\label{goals}

Tem-se por objetivo geral deste trabalho rq

\begin{outline}

    \1 rq1

    \1 rq2

\end{outline}

Dessa maneira, espera-se ...

\section{Referencial Teórico}
\label{bw}

\subsection{Machinations}
\label{bw:machinations}

\section{Trabalhos Relacionados}
\label{rw}

\section{Metodologia}
\label{m}

\subsection{Modelagem da sintaxe Machinations}
\label{m:modeling}
Os elementos descritos na seção \ref{bw:machinations} são definidos por \citeauthor{machinations} de maneira formal o suficiente a permitir uma simulação, mas não exatamente adequada para a implementação de um MDP no contexto de aprendizado por reforço. Assim, essa subseção se dedica à redefinição formal -- a algum nível de abstração -- dos elementos que compõem um diagrama de Machinations.

\subsubsection{Nós}
\textbf{Pools}, conforme descrito na seção \ref{bw:machinations}, são representadas como vértices $v \in V$. Definimos sua função de estado $x_{v,r}: V \times R \rightarrow \mathbb{R}$ como o número de recursos do tipo $r$ no nó $v$ e a função $m_{v}: V \rightarrow \{0,1,2\}$ como o modo de operação do nó entre passivo, automático ou interativo.

\textbf{Gates} são (...)

\subsubsection{Conexões}
Conexões são representadas como arestas $e \in E$, tal que $E$ é a união de uma série de subconjuntos que representam cada um dos diferentes tipos de conexões descritas na seção \ref{bw:machinations}.

\begin{align*}
    E &= E^R \cup E^S\\
    E^R &= \bigcup_{r \in R} E^r\\
    E^S &= E^{L}, E^{N}, E^{T}, E^{A}
\end{align*}

Essa composição, apesar de extensa, define toda a taxonomia das conexões definidas por \citeauthor{machinations}.

Conexões de recursos são representadas pelo subconjunto $E^R$ como a união das conexões para cada tipo de recurso. Uma conexão de recurso específica $e \in E^R$ é da forma $(u \rightarrow v)$ e tem definida para si uma função de taxa $T_e$ que define o fluxo de recursos do nó $u$ para o nó $v$ em um instante de tempo.

O conjunto $E^S$ define as conexões de estado. Essas, por sua vez, se dividem em: (...)

\subsubsection{Dinâmica}


\subsection{Implementação do problema}
\label{m:mdp}
Observabilidade parcial...

\subsection{Plano de Experimentos}
\label{m:experiments}

\printbibliography


\end{document}

